\documentclass{scrartcl}
\usepackage[utf8]{inputenc}
\usepackage[french,english]{babel}
\usepackage[scale=0.8]{geometry}
\usepackage[T1]{fontenc}
\usepackage{tikz}
\usepackage{graphicx}
\usepackage{wallpaper}

\title{Cahier des charges}
\author{Nathan Calvarin et Maxime Delin}
\subtitle{Tower Defense}
\date{\today}


\begin{document}
\selectlanguage{french}
\URCornerWallPaper{0.23}{/home/funeralworks/Documents/ENIB/ENIB-cartouche-cmjn.png}
\maketitle
\tableofcontents
\newpage

\section{Objectifs}
	\subsection{Description générale}
	Dans le cadre de notre cours de Méthode de Développement nous souhaitons recréer un jeu dans l'esprit d'un \emph{Tower Defense}.
	\subsection{Contexte}
\section{Expression du besoin}
\emph{Tower Defense} est un jeu pour un joueur dans lequel celui-ci doit disposer des tours dans le but de tuer des vagues de monstres. A chaque monstre éliminé revient au joueur une somme d'argent qui lui permet de racheter de nouvelles tours. 
	\subsection{Règles du jeu}
	\begin{itemize}
	\item Le but est d'éliminer les vagues de monstres à l'aide de tours disposée par l'utilisateur.
	\item Il y a un joueur : l'utilisateur.
	\item L'ordinateur envoie des monstres par vague qui ont pour but de traverser la \emph{map}.
	\item Le jeu débute avec une \emph{map} vide.
	\item Le joueur dispose d'une somme d'argent de départ et un certain nombre de vies.
	\item A chaque fois qu'un monstre atteint le bout de la \emph{map} le joueur perd une vie.
	\item Lorsqu'une tour est placée, le joueur peut choisir de l'améliorer.
	\end{itemize}
	\subsection{L'interface utilisateur}
		\subsubsection{Visuel}
		Pendant la partie, l'utilisateur voit la \emph{map}, son nombre de vie, son nombre argent, les vagues à venir, les tours qu'il peut choisir de disposer.
		
		\noindent Un écran de sortie affiche la mention \emph{Gagné} ou \emph{Perdu}, et le score du joueur, 
		\subsubsection{Interaction}
		L'utilisateur interagit uniquement avec le clavier : il se sert des flèches du clavier et des touches \emph{f}, \emph{g}, \emph{p} et \emph{espace}.

	\subsection{Manuel utilisateur}
	\begin{itemize}
		\item \textbf{Lancer le jeu} : \$python towerdefense.py
		\item \textbf{Saisir son nom} : taper son nom puis sur \emph{Entrée}
		\item \textbf{Choisir un niveau} : avec les flèches du clavier.
		\item \textbf{Jouer} : séletionner type de tour à l'aide des touches du clavier, poser tour en appuyant sur \emph{espace}, sélectionner tour déjà poser pour l'améliorer
		\item \textbf{Types de tour} : 
			\begin{itemize}
				\item \textbf{Tour feu} : l'utilisateur appuie sur la touche \emph{f} pour sélectionner ce type de tour.
				\item \textbf{Tour glace} : l'utilisateur appuie sur la touche \emph{g} pour sélectionner ce type de tour.
				\item \textbf{Tour poison} : l'utilisateur appuie sur la touche \emph{p} pour sélectionner ce type de tour.
			\end{itemize}
	\end{itemize}
	\subsection{Contraintes techniques}
	\begin{itemize}
   \item Le logiciel est associé à un cours, il doit donc fonctionner sur les machines de TP de l'ENIB pour que es élèves puissent le tester.
   \item Le langage utilisé en cours est Python. Le développement devra donc se faire en python.
   \item Les notions de programmation orientée objet n'ayant pas encore été abordées, le programme devra essentiellement s'appuyer sur le paradigme de la programmation procédurale.
   \item Le logiciel devra être réalisé en conformité avec les pratiques préconisées en cours de MDD: barrière d'abstraction, modularité, unicode, etc...
   \item L'interface sera réalisée en mode texte dans un terminal.
  \end{itemize}
	\subsection{Scénario d'utilisation}
		\subsubsection{S0: Scénario principal}
		\begin{tikzpicture}[node/.style={fill=green,draw=black,text=black,rounded corners},
					->,>=latex]

			\node[node] (1) at (0,5) {Lancer le jeu};
			\node[node] (2) at (1,4) {Saisir nom};
			\node[node] (3) at (2,3) {Choisir niveau};
			\node[node] (4) at (3,2) {S1 : Jouer manche};
			\node[node] (5) at (4,1) {Voir le score};

			\draw[->] (1) -- (2);
			\draw[->] (2) -- (3);
			\draw[->] (3) -- (4);
			\draw[->] (4) -- (5);
		\end{tikzpicture}




		\subsubsection{S1: Débuter une manche}
		\begin{tikzpicture}[node/.style={fill=green,draw=black,text=black,rounded corners},
					->,>=latex]

			\node[node] (1) at (0,5) {Débuter manche};
			\node[node] (2) [below of=1] {Voir la \emph{map}};
			\node[node] (3) [below of=2] {Selectionner case};
			\node[node] (4) [below right of=3] {Placer tour};
			\node[node] (5) [below of=4] {Niveau suivant};
			\node[node] (fin de manche) [below of=5,yshift=-1cm] {Fin de manche};
			\node[node] (se deplacer) [right of=2,xshift=1.5 cm] {Se déplacer};
			\node[node] (gagner) [above of=se deplacer] {\emph{Gagner}};
			\node[node] (ameliorer tour) [left of=3,yshift=-1cm, xshift=-3 cm] {S2 : Améliorer tour};
			\node[node] (type tour) [right of=5,xshift=4 cm] {Type de tour: appuie sur \emph{f},\emph{p} ou \emph{g}};


			\draw (1) -- (2);
			\draw (2) -- (3);
			\draw (3) -- (4);
			\draw (4) -- (5);
			\draw (5) -- node[right]{plus de vie} (fin de manche);
			\draw (4.east) to[out=90,in=0] (3.east);
			\draw (2) -- (se deplacer);
			\draw (se deplacer) -- (gagner);
			\draw (4.east) to[out=90,in=0] (se deplacer);
			\draw (3.south west) to[out=-90,in=90] (ameliorer tour.north);
			\draw (4.east) to[out=-90,in=180] (type tour.west);
			\draw (ameliorer tour.north) to[out=90,in=180] (1,7) to[out=0,in=90] (se deplacer.east);
		\end{tikzpicture}

		\subsubsection{S2 : Améliorer tour}
		\begin{tikzpicture}[node/.style={fill=green,draw=black,text=black,rounded corners},
					->,>=latex]
			\node[node] (selectionner case) at (0,0) {Selectionner tour};
			\node[node] (programme) [below of=selectionner case] {Le programme demande à chaque fois que l'on sélectionne une tour si on veut l'améliorer};
			\node[node] (oui) [below left of=programme,xshift=-3cm,yshift=-1 cm] {Touche \emph{o} : oui} ;
			\node[node] (non) [below right of=programme,xshift=3cm,yshift=-1 cm] {Touche \emph{n} : non} ;
			\node[node] (tour amelioree) [below of=oui] {La tour selectionnée est améliorée};

			\draw (selectionner case) -- (programme);
			\draw (programme) to[out=-90,in=90] (oui);
			\draw (programme) to[out=-90,in=90] (non);
			\draw (oui) to[out=-90,in=90] (tour amelioree);

		\end{tikzpicture}


 \section{Analyse du besoin:}
	\subsection{Fonctionnalités}
	  \begin{itemize}
	  \item F1: Nommer le joueur
	  \item F2: Choisir le niveau
	  \item F3: Jouer une partie
	
	  \begin{itemize}
	      \item F3.1: Joueur un niveau
	      \begin{itemize}
		  \item F3.1.1 Afficher le jeu
		    \begin{description}
		      \item{-} map
		      \item{-} nom
		      \item{-} niveau
		      \item{-} score
		      \item{-} case sélectionnée
		      \item{-} nombre de monstres restants
		      \item{-} différentes tours disponibles
		      \item{-} argent
		    \end{description}
		  \item F3.1.2 Sélectionner une tour
		  \item F3.1.3 Se déplacer dans la map
		  \item F3.1.4 Placer une tour
		  \item F3.1.5 Améliorer une tour
		  \item F3.1.6 Finir manche
		  \newline
		\end{itemize}
		
	      \item F3.2 Finir partie
		\begin{itemize}
		\item F3.2.1 Afficher le résultat
		\item F3.2.2 Quitter
		\end{itemize}
	  \end{itemize}
	\end{itemize}
	\subsection{Critères de validité et de qualité}
		\subsubsection{Validation}
		Le logiciel sera validé de la manière suivante:
  			\begin{itemize}
  			 	\item Le code doit s'exécuter correctement en suivant les instructions livrées avec le logiciel.
  			 	\item L'utilisation du logiciel permettra de constater que les fonctionnalités ont été bien implémentées.
  			\end{itemize}
		\subsubsection{Qualité}
		Différents critères permettront d'évaluer la qualité du jeu:
  			\begin{itemize}
  			 	\item La jouabilité: l'interface devra être suffisamment ergonomique pour permettre au joueur d'enchainer rapidement les manches.
  			 	\item La robustesse.
  			 	\item Le respect des méthodes de conception et de codage données en cours de MDD.
  			\end{itemize}
  \begin{flushleft}
		\subsubsection{Importance des fonctionnalités}
0: Indispensable

1: Forte valeur ajouté au projet

2: Optionnelle
  \newline
  
  \begin{tabular}{|l|c|}
    
    \hline
    F1: Nommer le joueur & 2 \\
    \hline
    F2: Choisir le niveau & 1 \\
    \hline
    F3: Jouer une partie & 0 \\
    \hline
    F3.1: Jouer un niveau & 0 \\
    \hline
    F3.1.1: Afficher le jeu & 0 \\
    \hline
    F3.1.2: Sélectionner une tour & 0 \\
    \hline
    F3.1.3: Se déplacer sur la map & 0 \\
    \hline
    F3.1.4: Placer une tour & 0 \\
    \hline
    F3.1.5: Améliorer une tour & 1 \\
    \hline
    F3.1.6: Fin du niveau & 0 \\
    \hline
    F3.2: Fin de la partie & 0 \\
    \hline
    F3.2.1: Afficher le résultat & 1 \\
    \hline
    F3.2.2: Quitter & 0 \\
    \hline

   \end{tabular}
  
    \section{Livrables}
	  \subsection{Échéancier}
  1\up{ère} semaine: cahier des charges
  
  2\up{ème} semaine: conception
  
  7\up{ème} semaine: P1 + version finale
  \newline
  Les livrables seront envoyé par mail en version électronique.
  
  Les documents texte seront au formats \emph{.pdf}
	  \subsection{Description des livrables}
		    \subsubsection{CDC: Cahier des charges}
		      Expression et analyse du besoin.
		      
		      Fichier: \emph{towerDefense\_delinCalvarin-CdC.pdf}
		    \subsubsection{C1: Document de conception v1.0}
		      Fichier \emph{towerDefense\_delinCalvarin-Conception.pdf}
		    \subsubsection{P1: Prototype P1}
		      Ce prototype porte essentiellement sur la création de la map et sur l'affichage.
		      
		      Mise en oeuvre des fonctionnalités: F1, F2, F3.1.1, F3.1.2, F3.1.3, F3.1.4, F3.1.5
		      
		      Livré dans une archive au format \emph{.zip} ou \emph{.tgz}
		      
		      Contient un manuel d'utilisation dans le fichier \emph{readme.txt}
		     \subsubsection{P2: Prototype P2}
		      Ce prototype réalise toutes les fonctionnalités.
		      
		      Ajout à P1 des fonctionnalités F3.1.6, F3.2
		      
		      Livré dans une archive au format \emph{.zip} ou \emph{.tgz}
		      
		      Contient un manuel d'utilisation dans le fichier \emph{readme.txt}
		     \subsubsection{VF: Version finale}
		      Archive finale au format \emph{.zip} ou \emph{.tgz}
		      
		      La version finale contient toutes les versions des documents: cahier des charges, conception, et les deux prototypes.
  
\end{flushleft}

\end{document}
\end{document}