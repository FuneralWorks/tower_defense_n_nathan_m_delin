\documentclass{article}
\usepackage[utf8]{inputenc}
\usepackage[french,english]{babel}
\usepackage[scale=0.8]{geometry}
\usepackage[T1]{fontenc}
\usepackage{tikz}
\usepackage{graphicx}
\usepackage{wallpaper}
\usepackage{alltt}

\title{Conception}
\author{Maxime Delin et Nathan Calvarin}
\date{\today}

\begin{document}

    \maketitle
    \newpage
    \tableofcontents
    \newpage
    
    \section{Rappel du cahier des charges}
        \subsection{Contraintes techniques}
        \begin{itemize}
        \item Le logiciel est associé à un cours, il doit fonctionner sur les machines de TP de l'ENIB pour que les élèves puissent les tester.
        \item Le langage utilisé est Python. Le développement devra donc se faire en python.
        \item Les notions de programmation orientée objet n'ayant pas encore été abordées, le programme devra essentiellement s'appuyer sur le paradigme de la programmation procédurale.
        \item Le logiciel devra être réalisé en conformité avec les pratiques préconisées en cours de MDD: barrière d'abstraction, modularité, unicode, etc.
        \item L'interface sera réalisée en mode texte dans un terminal.
        \end{itemize}
        \subsection{Fonctionnalités}
        \begin{itemize}
        \item F1: Nommer le joueur
        \item F2: Choisir le niveau
        \item F3: Jouer une partie
    
        \begin{itemize}
          \item F3.1: Joueur un niveau
          \begin{itemize}
          \item F3.1.1 Afficher le jeu
            \begin{description}
              \item{-} map
              \item{-} nom
              \item{-} niveau
              \item{-} score
              \item{-} case sélectionnée
              \item{-} nombre de monstres restants
              \item{-} différentes tours disponibles
              \item{-} argent
            \end{description}
          \item F3.1.2 Sélectionner une tour
          \item F3.1.3 Se déplacer dans la map
          \item F3.1.4 Placer une tour
          \item F3.1.5 Améliorer une tour
          \item F3.1.6 Finir manche
          \newline
        \end{itemize}
        
          \item F3.2 Finir partie
            \begin{itemize}
        \item F3.2.1 Afficher le résultat
        \item F3.2.2 Quitter
            \end{itemize}
            \end{itemize}
            \end{itemize}
        \subsection{P1: Prototype P1}
        Ce prototype porte essentiellement sur la création de la map et sur l'affichage.
        
        Mise en oeuvre des fonctionnalités: F1, F2, F3.1.1, F3.1.2, F3.1.3, F3.1.4, F3.1.5
        
        Livré dans un archive au format \emph{.zip} ou \emph{.tgz}
        
        Contient un manuel d'utilisation dans le fichier \emph{readme.txt}
        \subsection{P2: Prototype P2}
        Ce prototype réalise toutes les fonctionnalités.
        
        Ajout à P1 des fonctionnalités F3.1.6, F3.2
        
        Livré dans un archive au format \emph{.zip} ou \emph{.tgz}
        
        Contient un manuel d'utilisation dans le fichier \emph{readme.txt}
        \newpage
        
    \section{Principes des solutions techniques adoptées}
        \subsection{Langage}
        Conformément aux contraintes énoncées dans le cahier des charges, le codage est réalisé
avec langage python. Nous choisissons la version 2.7.5

        \subsection{Architecture du logiciel}
        Nous mettons en oeuvre le principe de la barrière d'abstraction. Chaque module correspond
à un type de donnée et fournit toutes les opérations permettant de le manipuler de manière
abstraite.

        \subsection{Interface utilisateur}
        L'interface utilisateur se fera via un terminal de type linux.
Nous reprenons la solution donnée en cours de MDD en utilisant les modules :
\emph{termios},\emph{sys}, \emph{select}.

            \subsubsection{Boucle de simulation}
            Le programme mettra en oeuvre une boucle de simulation qui gèrera l'affichage et les
événements clavier.

            \subsubsection{Affichage}
            L'affichage se fait en communicant directement avec le terminal en envoyant des chaînes de
caractères sur la sortie standard de l'application.

            \subsubsection{Gestion du clavier}
            L'entrée standard est utilisé pour détecter les actions de l'utilisateur.
Le module \emph{tty} permet de rediriger les événements clavier sur l'entrée standard.
Pour connaître les actions de l'utilisateur il suffit de lire l'entrée standard.

            \subsubsection{Image ascii-art}
            Pour dessiner certaines parties de l'interface nous utilisons des « images ascii ».
Dans l'idée de séparer le code et les données, les différentes images ASCII seront stockées
dans des fichiers textes : \emph{blalalalal.txt}, \emph{bkkgjmg.txt} .......

        \subsection{Map, tours}
    \section{Analyse d conception}
        \subsection{Analyse noms/verbes}
        \begin{description}
            \item[Verbes:] \hfill \\
            nommer,choisir,afficher,déplacer,placer,améliorer,finir,quitter
            \item[Noms:] \hfill \\
            jouer,niveau,nom,map,monstre,curseur,tour,argent,case selectionnée,nombre de monstres restant
           
            \end{description}
        \begin{itemize}

        \item[Verbes:]nommer,choisir,afficher,déplacer,placer,améliorer,finir,quitter
        \item[Noms:]jouer,niveau,nom,map,monstre,curseur,tour,argent,case selectionnée,nombre de monstres restant
        \end{itemize}
        \subsection{Type de donnée}

        \subsection{Dépendance entre modules}
        \subsection{Analayse descendante}
            \subsubsection{Arbre principal}
            \begin{alltt}
            \input{arbre_principal.txt}
            \end{alltt}
            \subsubsection{Arbre affichage}
            \subsubsection{Arbre interaction}
    \section{Description des fonctions}
        \subsection{Programme principal: Main.py}
        \subsection{Game.py}
        \subsection{Map.py}
    \section{Calendrier et suivi de développement}
        \subsection{P1}
            \subsubsection{Fonctions à développer}
            \subsubsection{Autre}
        \subsection{P2}
            \subsubsection{Fonctions à développer}
\end{document}